\documentclass{sigchi}

% Load basic packages
\usepackage{balance}  % to better equalize the last page
\usepackage{graphics} % for EPS, load graphicx instead
\usepackage{times}    % comment if you want LaTeX's default font
\usepackage{url}      % llt: nicely formatted URLs
\usepackage{caption}
\usepackage{subcaption}
\usepackage{float}
\usepackage{pdfpages}

% llt: Define a global style for URLs, rather that the default one
\makeatletter
\def\url@leostyle{%
  \@ifundefined{selectfont}{\def\UrlFont{\sf}}{\def\UrlFont{\small\bf\ttfamily}}}
\makeatother
\urlstyle{leo}


% To make various LaTeX processors do the right thing with page size.
\def\pprw{8.5in}
\def\pprh{11in}
\special{papersize=\pprw,\pprh}
\setlength{\paperwidth}{\pprw}
\setlength{\paperheight}{\pprh}
\setlength{\pdfpagewidth}{\pprw}
\setlength{\pdfpageheight}{\pprh}

% Make sure hyperref comes last of your loaded packages,
% to give it a fighting chance of not being over-written,
% since its job is to redefine many LaTeX commands.
\usepackage[pdftex]{hyperref}
\hypersetup{
pdftitle={Approximator - An infrastructure for presence and interruptibility-sharing},
pdfauthor={Kristian S. M. Andersen, Anders Bech Mellson and Mads Daniel Christensen},
pdfkeywords={spce, gse, presence, awareness},
bookmarksnumbered,
pdfstartview={FitH},
colorlinks,
citecolor=black,
filecolor=black,
linkcolor=black,
urlcolor=black,
breaklinks=true,
}

% create a shortcut to typeset table headings
\newcommand\tabhead[1]{\small\textbf{#1}}


% End of preamble. Here it comes the document.
\begin{document}

\title{Approximator}
\subtitle{An infrastructure for presence and interruptibility-sharing}
\numberofauthors{3}
\author{
  \alignauthor Anders Bech Mellson\\
    \email{anbh@itu.dk}\\
  \alignauthor Kristian S. M. Andersen\\
    \email{ksma@itu.dk}\\
  \alignauthor Mads D. Christensen\\
    \email{mdch@itu.dk}\\
}

\maketitle

\begin{abstract}

\end{abstract}

\keywords{
  Global Software Engineering, Presence Awareness, Interruptibility, Social Contracts of Human Interactions
}

\terms{
  Documentation, Theory
}

\section{Introduction}
Globalization is an economical and societal trend that has pushed industries to move from local to global markets.
Working in a global setting requires practitioners to work in distributed arrangements.

Paraphrasing Herbsleb \footnote{Herbsleb, James D. (2007). Global Software Engineering: The Future of Socio-technical Coordination}, many of the mechanisms that work correctly in a co-located setting are absent or disrupted in a distributed arrangement.
Different approaches have been investigated to improve the awareness of the working context that a member of a virtual team has; nonetheless, information like the presence of virtual team members, trivial in a co-located setting, represent an interesting open area of investigation.
% TODO hvem har sagt ovenstående? Er det fra Herbsleb? Tror vi skal specificere lidt mere end bare skrive different approaches have been investigated.

One fundamental difference between a co-located arrangement and a distributed one is the lack of presence awareness.
The ability to assess how interruptible another person is, becomes very hard when you are not in the same room.
When you are in the near proximity of another person it is fairly easy to guess the interruptibility of that person.
We use the interruptibility assessment to facility behavior that we consider socially acceptable.
If you are in a distributed setting you are dependent on tool support from computer and communication systems.
These systems are largely unaware of the social contracts of human interactions.

Research by the Human Computer Interaction Institute at Carnegie Mellon University \cite{Fogarty:2005:PHI:1057237.1057243} shows that you can detect the interruptibility of a person using sensor inputs.
In their research they find that you can predict interruptibility with sensor inputs to an accuracy of 68\%.
The researchers note that their results could ``motivate the development of systems that use these models to negotiate interruptions at socially appropriate times.''

We will try to build upon the research done by Fogarty etc.
The work presented in this paper represents two primary contributions.
First, we demonstrate an implementation of a presence and interruptibility-sharing infrastructure for global software engineers.
Second, we demonstrate a possible application. This is used to test and evaluate the infrastructure by implementing enough sensors to provide accurate interruptibility information.

\section{Implementation}
\subsection{Infrastructure}
The system needs to be highly scalable to support many users.
For this reason we will create a system based on the Actor model by Carl Hewitt \footnote{Carl Hewitt; Peter Bishop; Richard Steiger (1973). A Universal Modular Actor Formalism for Artificial Intelligence}.

\section{Related work}
% TODO vi skal have en reference ind her så det ikke bare bliver en påstand!
The idea of context aware systems started at Xerox PARC through the idea of smart spaces, smart rooms and other context aware technologies.
These context aware systems made it possible to make a system react to a given situation.
For instance if a person walks into a smart room, his work related documents would be accessible in that room automatically.

NooSphere \cite{houben2013noosphere} is a service based activity-centric infrastructure that supports the development and deployment of distributed interactive systems.
It conceptualizes activity as its base of context, meaning its first class object is an activity that is minimally composed of users, meta-information, actions and resources related to that activity.
NooSphere enables the development of context aware systems both through its infrastructure and because it has a built in discovery system that makes it easy to sense other devices around it.

Another example of an infrastructure that supports context aware applications is Ocon \cite{ocon}.
Ocon is a context aware system built to emulate and enhance the properties of a scrum-board.
Even though the papers proof-of-concept and motivation was the scrum-board, their architecture could be used for other context aware applications.
Ocon is built in .NET and consists of three components (i) central (center of infrastructure, decides when and how to notify the clients), (ii) client (acts as actuators, notified when a specific event occur) and (iii) widget (translates and sends sensory input to central).

For this specific project we feel that NooSphere is too extensive for our needs and the Ocon paper have not indicated that Ocon is scalable.
Because of this we have decided to create our own infrastructure.

For this infrastructure we will be using the actor model \cite{hewitt1973universal}.
The actor model is a way to model computer programs where everything is split up into the smallest possible unit of work.
These units are encapsulated in an actor.
And the actors communicate by sending messages to each other.
Actors does not share any state making it possible to run actors in parallel and even on different machines.
This makes the actor model a good candidate for building highly scalable applications.

For evaluation we will use Gutwins awareness framework and the C3 model as points of measurement.

\section{Technical Implementation}


\section{Discussion}

\section{Conclusion}

\balance
\bibliographystyle{acm-sigchi}
\bibliography{ubicomp}

% \section{Appendix}
\end{document}
