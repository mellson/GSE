\documentclass{sigchi}

% Load basic packages
\usepackage{balance}  % to better equalize the last page
\usepackage{graphics} % for EPS, load graphicx instead
\usepackage{times}    % comment if you want LaTeX's default font
\usepackage{url}      % llt: nicely formatted URLs
\usepackage{caption}
\usepackage{subcaption}
\usepackage{float}
\usepackage{pdfpages}

% llt: Define a global style for URLs, rather that the default one
\makeatletter
\def\url@leostyle{%
  \@ifundefined{selectfont}{\def\UrlFont{\sf}}{\def\UrlFont{\small\bf\ttfamily}}}
\makeatother
\urlstyle{leo}


% To make various LaTeX processors do the right thing with page size.
\def\pprw{8.5in}
\def\pprh{11in}
\special{papersize=\pprw,\pprh}
\setlength{\paperwidth}{\pprw}
\setlength{\paperheight}{\pprh}
\setlength{\pdfpagewidth}{\pprw}
\setlength{\pdfpageheight}{\pprh}

% Make sure hyperref comes last of your loaded packages,
% to give it a fighting chance of not being over-written,
% since its job is to redefine many LaTeX commands.
\usepackage[pdftex]{hyperref}
\hypersetup{
pdftitle={Presence Awareness in Global Software Engineering},
pdfauthor={Kristian S. M. Andersen, Anders Bech Mellson and Mads Daniel Christensen},
pdfkeywords={spce, gse, presence, awareness},
bookmarksnumbered,
pdfstartview={FitH},
colorlinks,
citecolor=black,
filecolor=black,
linkcolor=black,
urlcolor=black,
breaklinks=true,
}

% create a shortcut to typeset table headings
\newcommand\tabhead[1]{\small\textbf{#1}}


% End of preamble. Here it comes the document.
\begin{document}

\title{Presence Awareness in Global Software Engineering}

\numberofauthors{3}
\author{
  \alignauthor Anders Bech Mellson\\
    \email{anbh@itu.dk}\\
  \alignauthor Kristian S. M. Andersen\\
    \email{ksma@itu.dk}\\
  \alignauthor Mads D. Christensen\\
    \email{mdch@itu.dk}\\
}

\maketitle

\keywords{
  Activity Monitoring
}

\terms{
  Project proposal, Documentation, Theory
}

\section{Background}

Nowadays, globalisation is an economical and societal trend that has pushed industries to move from local to global markets requiring practitioners to work more and more in distributed arrangements.  However, this rapid shift of practice has not been followed timely by organisational changes. In fact, a fundamental difference between a co-located arrangement and a distributed one is represented by the lack of physical presence.  Often underestimated, this intrinsic characteristic has severe repercussions, which need to be properly addressed for the distributed cooperation to succeed.
Paraphrasing Herbsleb \footnote{Herbsleb, James D. (2007). Global Software Engineering: The Future of Socio-technical Coordination}, many of the mechanisms that function in a co-located setting are absent or disrupted in a distributed project. Different approaches have been investigated to improve the awareness of the working context that a member of a virtual team has; nonetheless, information like the presence of virtual team members, trivial in a co-located setting, represent an interesting open area of investigation.

\section{Idea}

We wish to create a presence information infrastructure for global software engineers. The central idea behind the project is focused on proposing a solution to the context awareness problem described in the background. Our initial conceptualisation of a possible implementation of such solution considers a modular system able to gather presence information from inputs sources (sensors), which would them be exposed via a REST API.
Since there could be many workers using this system, each having many sensors and consumers (demonstrators), the system needs to be highly scalable. For this reason we will create a system based on the Actor model by Carl Hewitt \footnote{Carl Hewitt; Peter Bishop; Richard Steiger (1973). A Universal Modular Actor Formalism for Artificial Intelligence}.
The infrastructure holds the current state of each worker's presence, obtained through the sensors. We will use this information to create a number of demonstrators, showing possible applications translating the presence information into awareness and availability.

\section{Scenario}

In our scenario we look at a small danish Software Development company called Robocat located in central Copenhagen. The company has 11 employees where 7 employees work in the Copenhagen Office, 2 in Aarhus and 2 in San Francisco.

Due to the distributed nature of the company it is very hard to gain an overview of the people present at work and whether their presence is in the office or somewhere else. Workers are teamed up on projects that span multiple locations and often need to coordinate ad hoc meetings.

In this scenario the collective knowledge about coworkers presence could alleviate some of the issues that occur from the global distribution of teams. The teams could use our infrastructure to collect presence about all workers and feed it to demonstrators that would help the team to gain better insights into the collective status of the team. The sensors that detect their presence is feeding the infrastructure with information through the REST API, describing their individual situation. These sensors could be:
\begin{itemize}
\item The proximity of a team member’s phone to their work computer.
\item An application installed on the team member’s computer that detects work-related activities.
\item A camera monitoring movement patterns in the room where the work is executed.
\item A heat sensor/thermal camera detecting movement in the work area.
\item A pressure plate on the team member’s chair.
\item A RFID chip that the team member must log in with.
\item A bot logged in to the teams work chat room watching team members online status
\end{itemize}

Each of the team members could have an application on their workstation (one of the demonstrators), exposing availability information of the other team members. The program calls the service through the REST API and retrieves information about the other team members. Based on this information, it determines whether each team member is present and available. Examples of demonstrators in this scenario could be:

\begin{itemize}
\item An application installed on the team member’s mobile device that displays the status of all team members.
\item USB toy figures conveying presence information of team members.
\item A service that detects when the team member tries to contact another team member. The service will then inform the interrupter about the status of the interruptee.
\item An application that can visualize presence information over time and use it for statistical purposes.
\item A large wall mounted display that can convey information about many team members.
\end{itemize}

All of this information can be accessed by anyone within the company, and can be used to get a quick overview of how the company’s resources are being used. It can also be used by a developer inside the company to see where another employee in charge of a given area is currently located and if that person is busy.

\section{Requirements}

Server for the infrastructure. We will use Azure, Digital Ocean or similar.
Sensors for the presence information gathering. We will consider using some of the following; smartphones, heat sensors, cameras, microphones, RFID sensors, pressure sensors, software sensors.
The infrastructure should be able to handle data coming from multiple sources even when carrying conflicting information.
Demonstrators showcasing the infrastructure. Such as mobile devices, screens, holographic displays, speakers, indicator lights, wall-mounted displays.

\section{Supervisors}

Paolo Tell and Jakob Bardram

\section{Foreseen Outcome}

The outcome of this project will end up with the following products.
\begin{enumerate}
\item (prototype) infrastructure. The server software system supporting the entire project
\item (prototype) REST APIs. The software exposing the functionalities provided by 1.
\item (prototype) demonstrators e.g. actuators that uses the information from the infrastructure.
\item (report) infrastructure design. Internal document detailing all design decisions, diagrams, etc related to 1. This document will be used during the execution of 1 to collect all the requirements and create the design of the infrastructure
\item (report) APIs documentation. Formal documentation of the functionalities of 1. exposed
\item (report) input sources. Internal document capturing all the discussions including rationale related to the HW/SW considered for sensing the presence.  The document will contain also a summarizing matrix.
\item (handout) final report. Formal document for the course in the format of a Ubicomp paper.
\end{enumerate}

\section{Risk management}

If we run into an unexpected complication we have assessed that the minimum work we willhandin is the infrastructure prototype and the Ubicomp report.

\section{Logistic}

\begin{itemize}
\item We have a weekly meeting on every tuesday. During this meeting:
    \begin{itemize}
    \item Every team member is available on Slack and Skype.
    \item We discuss the work progress during the last week.
    \item Assess when we have the need for an alignment meeting with our supervisor.
    \item Follow up on the schedule.
    \end{itemize}
\item We have a shared folder on Google Drive which is accessible by team members and the supervisor.
\item We use GitHub to version and ensure our prototypes and report.
\end{itemize}

\section{Plan with work packages}

\begin{itemize}
\item Overall Information
    \begin{itemize}    
    \item start week 37 - 09.09.14
    \item duration 14 weeks - 1 week vacation
        \begin{itemize}
        \item Infrastructure 5 weeks
        \item Demonstrators 4 weeks
        \item Report 4 weeks
        \end{itemize}
    \end{itemize}
\item Infrastructure prototype
    \begin{itemize}
    \item This package comprises the design and implementation of the infrastructure. Following the tasks and deliverables.
    \item start week 37 - 09.09.14
    \item duration 4 weeks
    \item Tasks
        \begin{itemize}
        \item Requirement identification
        \item High level design
        \item Technical description
        \item Implementation
        \item Test
        \end{itemize}
    \item Deliverables
        \begin{itemize}
        \item Initial design
        \item Technical description
        \item Final design
        \end{itemize}
    \end{itemize}
\item Demonstrator prototypes
    \begin{itemize}
    \item Design and implementation of a few demonstrators that will use the infrastructure to create awareness and availability information.
    \item start week 43
    \item duration weeks
    \item Tasks
        \begin{itemize}
        \item Brainstorm potential demonstrators
        \item Select and decide which demonstrators to create
        \item Technical description
        \item Implementation
        \item Test
        \end{itemize}
    \item Deliverables
        \begin{itemize}
        \item Initial selection of demonstrators with descriptions
        \item Design of each demonstrator
        \item Technical description of each demonstrator
        \item How-to guide for each implemented demonstrator
        \end{itemize}
    \end{itemize}
\item Report
    \begin{itemize}
    \item Report written using the Ubicomp template describing the infrastructure, looking at similar work, evaluation of the infrastructure and showcasing a few demonstrators.
    \item start week 48
    \item duration weeks
    \item Tasks
        \begin{itemize}
        \item Write outline
        \item Find related work
        \item Create figures and technical illustrations
        \end{itemize}
    \item Deliverables
        \begin{itemize}
        \item Initial report outline
        \item Figures and technical illustrations
        \item Final report outline
        \item Report drafts
        \item Final report
        \end{itemize}
    \end{itemize}
\end{itemize}

\section{Related work}

J. Herbsleb. Global software engineering: The future of socio- technical coordination. 2007 Future of Software Engineering, 2007.

J. D. Hincapié-Ramos, S. Voida, and G. Mark, "A Design Space Analysis of Availability-sharing Systems," presented at the Proceedings of the 24th Annual ACM Symposium on User Interface Software and Technology, 2011.

I. Steinmacher, A. Chaves, and M. Gerosa, “Awareness Support in Distributed Software Development: A Systematic Review and Mapping of the Literature,” Computer Supported Cooperative Work (CSCW), vol. 22, no. 2, pp. 113–158, 2013.

S. Houben, S. Nielsen, M. Esbensen, and J. E. Bardram, “NooSphere: An Activity-centric Infrastructure for Distributed Interaction,” presented at the Proceedings of the 12th International Conference on Mobile and Ubiquitous Multimedia, New York, NY, USA, 2013, pp. 13:1–13:10.

Y. Dittrich, D. Randall, and J. Singer, “Software Engineering as Cooperative Work,” Computer Supported Cooperative Work (CSCW), vol. 18, no. 5, pp. 393–399, 2009.

K. Schmidt, “Cooperative work and its articulation: requirements for computer support,” Le Travail Humain, pp. 345–366, 1994.

Jacob B. Cholewa and Mathias K. Pedersen, ocon "a context aware framework”, 2014


%\balance
%\bibliographystyle{acm-sigchi}
%\bibliography{ubicomp}

% \section{Appendix}
\end{document}
